\documentclass[a5paper,10pt,twoside]{article}

\usepackage{hyperref}

\usepackage[disablejfam]{luatexja}
\usepackage{luatexja-fontspec}
\usepackage{luatexja-ruby}

\usepackage{expl3}
\usepackage{chngcntr}
\usepackage{adjustbox}
\usepackage{calc}
\usepackage{fancyhdr}
\usepackage{hanging}
\usepackage{ifthen}
\usepackage{mfirstuc}
\usepackage{multicol}
\usepackage{pdfcomment}
\usepackage{pxpic}
\usepackage{enumitem}
\usepackage{titlesec}
\usepackage{xparse}

\usepackage{newunicodechar}

\usepackage{geometry}

\setmainfont{Junicode}
\newunicodechar{Д}{{\setmainfont{EB Garamond}Д}}
\newunicodechar{х}{{\setmainfont{EB Garamond}х}}
\newunicodechar{ъ}{{\setmainfont{EB Garamond}ъ}}
\newunicodechar{ь}{{\setmainfont{EB Garamond}ь}}
\newunicodechar{М}{{\setmainfont{EB Garamond}М}}
\newunicodechar{О}{{\setmainfont{EB Garamond}О}}
\newunicodechar{Т}{{\setmainfont{EB Garamond}Т}}
\newunicodechar{г}{{\setmainfont{EB Garamond}г}}
\newunicodechar{л}{{\setmainfont{EB Garamond}л}}
\newunicodechar{н}{{\setmainfont{EB Garamond}н}}
\newunicodechar{о}{{\setmainfont{EB Garamond}о}}
\newunicodechar{р}{{\setmainfont{EB Garamond}р}}
\newunicodechar{у}{{\setmainfont{EB Garamond}у}}

%\setmainfont[Numbers={Lining,Monospaced}]{EB Garamond}
%\setsansfont{Noto Sans}
\setfontfamily\fSemanticLabel{Gentium Plus}[CharacterVariant={43:1}]
\setfontfamily\fReadingIndexLabel{EB Garamond}
\setfontfamily\fCyrilic{EB Garamond}
\setfontfamily\fGreek{EB Garamond}

% Noto
%
% \setmainjfont{Noto Serif CJK SC}[JFM=CCT]
% \setsansjfont{Noto Sans CJK SC}[JFM=CCT]
% 
% \newjfontfamily\fontMainJapn{Noto Serif CJK JP}[]
% \newjfontfamily\fontSansJapn{Noto Sans CJK JP}[]
% \newjfontfamily\fontMainKo{Noto Serif CJK K}[]
% \newjfontfamily\fontSansKo{Noto Sans CJK K}[]
% \newjfontfamily\fontMainHans{Noto Serif CJK SC}[JFM=CCT]
% \newjfontfamily\fontSansHans{Noto Sans CJK SC}[JFM=CCT]

% Source Han
% 
\setmainjfont{Source Han Serif SC}[
%  AltFont={
%    {Range={
%      "205E8, "201A2, "2090E, "20B9F,
%      "21484, "210E4, "21883, "21A25,
%      "224ED, "2274A, "22D67, "23154,
%      "24169, "25CD1, "25E24, "26855,
%      "2696F, "26C29, "278B3, "27E87,
%      "27FB7, "280B7, "284DC, "289E8,
%      "2B7CF, "2C1C7
%    },
%    Font=HanaMinB}
%  },
  JFM=CCT
]
\setsansjfont{Source Han Sans SC}[JFM=CCT]

\newjfontfamily\fontMainJapn{Source Han Serif}[]
\newjfontfamily\fontSansJapn{Source Han Sans}[]
\newjfontfamily\fontMainKo{Source Han Serif K}[
%  AltFont={
%    {Range={
%      "5F4D, "7926, "7AEB, "87E5, "89F5, "913A, "92C1
%    },
%    Font=SourceHanSerifJP}
%  }
]
\newjfontfamily\fontSansKo{Source Han Sans K}[]
\newjfontfamily\fontMainHans{Source Han Serif SC}[
%  AltFont={
%    {Range={
%      "205E8, "201A2, "20B9F, "2090E,
%      "21484, "210E4, "21883, "21A25,
%      "224ED, "2274A, "22D67, "23154,
%      "24169, "25CD1, "25E24, "26855,
%      "2696F, "26C29, "278B3, "27E87,
%      "27FB7, "280B7, "284DC, "289E8,
%      "2B7CF, "2C1C7
%    },
%    Font=HanaMinB}
%  },
  JFM=CCT
]
\newjfontfamily\fontSansHans{Source Han Sans SC}[JFM=CCT]

% Make Greek and Cyrillic non-CJK
\ltjsetparameter{jacharrange={-2}}

% Make en-dash non-CJK
\ltjdefcharrange{8}{`–,`‹,`›}

% Make super-/subscripts Latin
\ltjdefcharrange{1}{"2070-"209F}
\ltjsetparameter{alxspmode={`₁,inhibit}}
\ltjsetparameter{alxspmode={`₂,inhibit}}

\newcommand{\textJapn}[1]{{%
    \fontMainJapn%
    #1%
  }}
\newcommand{\textKor}[1]{{%
    \fontMainKo
    #1%
  }}
\newcommand{\textHant}[1]{{%
    \fontMainHans%
    #1%
  }}
\newcommand{\textHans}[1]{{%
    \fontMainHans%
    #1%
  }}

% When the current style is m,
% attempting to change to bx,
% use bx if availabke,
% otherwise use b.
\DeclareFontShapeChangeRule{m}{bx}{bx}{b}

\newlength{\EntryDescriptionLineLength}
\newlength{\EntryDescriptionLineHeight}
\newlength{\EntryDescriptionLineIndent}

\providecommand\phantomsection{}

\geometry{top=10mm,headheight=18pt,headsep=2pt,inner=20mm,outer=10mm,textheight=55em}

\ExplSyntaxOn

\tl_new:N \g_zqdict_entries_tl
\bool_new:N \g_zqdict_entries_incomplete_bool
\tl_new:N \g_zqdict_entries_archived_tl
\int_new:N \g_zqdict_entries_archived_page_int
\tl_new:N \g_zqdict_current_section_tl

\cs_set:Nn \_zqdict_xr_begin: {
  \tl_build_gbegin:N \g_zqdict_entries_tl
  \bool_gset_true:N \g_zqdict_entries_incomplete_bool
}
\cs_set:Nn \_zqdict_xr_section:nnnn {
  \tl_gset:Nn \g_zqdict_current_section_tl {{#1}{#2}{#3}{#4}}
}
\cs_set:Nn \_zqdict_xr_entry_begin:nn {
  \tl_build_gput_right:Nx \g_zqdict_entries_tl {{{begin}{\exp_not:n {#1}}{\exp_not:n {#2}}{\tl_use:N{\g_zqdict_current_section_tl}}}}
}
\cs_set:Nn \_zqdict_xr_entry_end:nn {
  \tl_build_gput_right:Nx \g_zqdict_entries_tl {{{end}{\exp_not:n {#1}}{\exp_not:n {#2}}{\tl_use:N{\g_zqdict_current_section_tl}}}}
}
\cs_set:Nn \_zqdict_xr_end: {
  \bool_if:NT \g_zqdict_entries_incomplete_bool {
    \tl_build_gend:N \g_zqdict_entries_tl
    \bool_gset_false:N \g_zqdict_entries_incomplete_bool
  }
}

\cs_set:Nn \_zqdict_entry_get_type:n {
  \tl_item:nn {#1} {1}
}
\cs_set:Nn \_zqdict_entry_get_headword:n {
  \tl_item:nn {#1} {2}
}
\cs_set:Nn \_zqdict_entry_get_page:n {
  \tl_item:nn {#1} {3}
}
\cs_set:Nn \_zqdict_entry_get_section:n {
  \tl_item:nn {#1} {4}
}

\cs_set:Nn \_zqdict_section_get_partlabel:n {
  \tl_item:nn {#1} {1}
}
\cs_set:Nn \_zqdict_section_get_number:n {
  \tl_item:nn {#1} {2}
}
\cs_set:Nn \_zqdict_section_get_options:n {
  \tl_item:nn {#1} {3}
}
\cs_set:Nn \_zqdict_section_get_label:n {
  \tl_item:nn {#1} {4}
}

\cs_generate_variant:Nn \_zqdict_entry_get_type:n { o , x , V , v }
\cs_generate_variant:Nn \_zqdict_entry_get_headword:n { o , x , V , v }
\cs_generate_variant:Nn \_zqdict_entry_get_page:n { o , x , V , v }
\cs_generate_variant:Nn \_zqdict_entry_get_section:n { o , x , V , v }
\cs_generate_variant:Nn \_zqdict_section_get_partlabel:n { o , x , V , v }
\cs_generate_variant:Nn \_zqdict_section_get_number:n { o , x , V , v }
\cs_generate_variant:Nn \_zqdict_section_get_options:n { o , x , V , v }
\cs_generate_variant:Nn \_zqdict_section_get_label:n { o , x , V , v }
\cs_generate_variant:Nn \tl_if_eq:nnTF { xnTF, fnTF }
\cs_generate_variant:Nn \tl_if_eq:nnT { xnT, fnT }


\AtBeginDocument {
  \iow_now:cx { @auxout } {
    \token_to_str:N \ExplSyntaxOn
    \token_to_str:N \_zqdict_xr_begin:
    \token_to_str:N \ExplSyntaxOff
  }
}
\NewDocumentCommand{\RecordSection}{O{} m O{} m}{
  \iow_shipout_x:cn { @auxout } {
    \token_to_str:N \ExplSyntaxOn
    \token_to_str:N \_zqdict_xr_section:nnnn { #1 } { #2 } { #3 } { #4 }
    \token_to_str:N \ExplSyntaxOff
  }
}
\NewDocumentCommand{\RecordEntryBegin}{m}{
  \iow_shipout_x:cn { @auxout } {
    \token_to_str:N \ExplSyntaxOn
    \token_to_str:N \_zqdict_xr_entry_begin:nn { #1 } { \int_eval:n { \value { page } } }
    \token_to_str:N \ExplSyntaxOff
  }
}
\NewDocumentCommand{\RecordEntryEnd}{m}{
  \iow_shipout_x:cn { @auxout } {
    \token_to_str:N \ExplSyntaxOn
    \token_to_str:N \_zqdict_xr_entry_end:nn { #1 } { \int_eval:n { \value { page } } }
    \token_to_str:N \ExplSyntaxOff
  }
}
\AtBeginDocument {
  \_zqdict_xr_end:
}

\str_new:N \l_zqdict_item_entry_type_str

\str_new:N \l_zqdict_current_part_str
\tl_new:N \l_zqdict_current_entry_tl
\str_new:N \l_zqdict_previous_section_str
\tl_new:N \l_zqdict_buffer_part_tl

\int_new:N \l_zqdict_alive_page_int
\int_new:N \l_zqdict_current_page_int

\cs_set:Nn \_zqdict_archive_items_until_page:n {
  \int_set:Nn \l_zqdict_alive_page_int {#1}
  \tl_log:n {Remove items till page \int_use:N \l_zqdict_alive_page_int}

  \bool_if:nTF {
    \int_compare_p:nNn {\g_zqdict_entries_archived_page_int + 10} < { \l_zqdict_alive_page_int }
  } {
    \tl_map_inline:Nn \g_zqdict_entries_tl {
      \int_set:Nx \l_zqdict_current_page_int {
        \_zqdict_entry_get_page:n {##1}
      }
      \int_compare:nNnTF {\l_zqdict_current_page_int} < {\l_zqdict_alive_page_int} {
        \tl_gput_left:Nn \g_zqdict_entries_archived_tl {{##1}}
        \tl_gset:Nx \g_zqdict_entries_tl {
          \tl_tail:N \g_zqdict_entries_tl
        }
      } {
        \tl_map_break:n {}
      }
    }
  } {
    \tl_map_inline:Nn \g_zqdict_entries_archived_tl {
      \int_set:Nx \l_zqdict_current_page_int {
        \_zqdict_entry_get_page:n {##1}
      }
      \int_compare:nNnTF {\l_zqdict_current_page_int} < {\l_zqdict_alive_page_int} {
        \tl_map_break:n {}
      } {
        \tl_gput_left:Nn \g_zqdict_entries_tl {{##1}}
        \tl_gset:Nx \g_zqdict_entries_archived_tl {
          \tl_tail:N \g_zqdict_entries_archived_tl
        }
      }
    }
  }

  \int_gset:Nn \g_zqdict_entries_archived_page_int {#1}
}
\cs_generate_variant:Nn \_zqdict_archive_items_until_page:n { o , x , V , v }
\cs_generate_variant:Nn \int_show:n { o , x }
\cs_generate_variant:Nn \int_set:Nn { No , Nx }

\bool_new:N \l_zqdict_header_render_elements_first_bool
\cs_set:Nn \_zqdict_header_render_elements:nn {
  \clist_if_in:nnTF {#1} {rangeElements} {
    \tl_if_empty:nF {#2} {
      \tl_if_single:nTF {#2} {
        \tl_head:n {#2}
      } {
        \tl_head:n {#2}
        --
        \tl_item:nn {#2} {-1}
      }
    }
  } {
    \bool_set_true:N \l_zqdict_header_render_elements_first_bool
    \tl_map_inline:Nn {#2} {
      %\tl_show:n {##1}
      \bool_if:NF \l_zqdict_header_render_elements_first_bool {
        \bool_set_false:N \l_zqdict_header_render_elements_first_bool
        \hspace{0.25em}
      }

      ##1

    }
    \bool_set_false:N \l_zqdict_header_render_elements_first_bool
  }
}
\cs_generate_variant:Nn \_zqdict_header_render_elements:nn { nV, xV, fV, xx }

\NewDocumentCommand{\ShowEntriesForPage}{m}{
  \_zqdict_archive_items_until_page:x {\int_eval:n {#1 - 1}}

  % \tl_show:x {
  %   Debug~print~at~p. \int_eval:n {#1}
  %   ^^J
  %   The~first~item: \tl_head:N \g_zqdict_entries_tl
  %   ^^J
  %   The~archived~items: \tl_use:N \g_zqdict_entries_archived_tl
  %   ^^J
  %   The~archived~page: \int_use:N \g_zqdict_entries_archived_page_int
  % }

  \int_set:Nn \l_zqdict_alive_page_int {#1}
  \str_clear:N \l_zqdict_previous_part_str
  \tl_clear:N \l_zqdict_current_entry_tl
  \tl_clear:N \l_zqdict_buffer_part_tl

  \tl_map_inline:Nn \g_zqdict_entries_archived_tl {
    \tl_if_eq:xnTF {\_zqdict_entry_get_type:n {##1}} {end} {
      \tl_map_break:n {}
    } {
      \tl_if_eq:xnT {\_zqdict_entry_get_type:n {##1}} {begin} {
        \tl_set:Nn \l_zqdict_current_entry_tl {##1}
        \tl_map_break:n {}
      }
    }
  }

  \tl_map_inline:Nn \g_zqdict_entries_tl {
    \str_set:Nx \l_zqdict_item_entry_type_str {
      \_zqdict_entry_get_type:n {##1}
    }

    \tl_if_eq:xnT {\_zqdict_entry_get_type:n {##1}} {begin} {
      \tl_set:Nn \l_zqdict_current_entry_tl {##1}
    }

    \tl_if_empty:NT \l_zqdict_current_entry_tl {
      \tl_show:x {
        Beginning~events~lack~at~p. \int_eval:n {\l_zqdict_alive_page_int}
        ^^J
        The~first~item: \tl_head:N \g_zqdict_entries_tl
        ^^J
        The~archived~items: \tl_use:N \g_zqdict_entries_archived_tl
        ^^J
        The~archived~page: \int_use:N \g_zqdict_entries_archived_page_int
        ^^J
        Trying~to~recover
      }
      \tl_set:Nn \l_zqdict_current_entry_tl {##1}
    }

    \tl_set:Nx \l_tmpa_tl {
      \_zqdict_entry_get_section:V \l_zqdict_current_entry_tl
    }
    \str_set:Nx \l_zqdict_current_section_str {
      \_zqdict_section_get_label:V \l_tmpa_tl
    }
    \int_set:Nx \l_zqdict_current_page_int {
      \_zqdict_entry_get_page:V \l_zqdict_current_entry_tl
    }
    \bool_if:nT {
      (
        \str_if_eq_p:Vn \l_zqdict_item_entry_type_str { begin }
        &&
        \int_compare_p:nNn {\l_zqdict_current_page_int} = { \l_zqdict_alive_page_int }
      ) || (
        \str_if_eq_p:Vn \l_zqdict_item_entry_type_str { end }
        &&
        \int_compare_p:n {
          \l_zqdict_current_page_int
          < \l_zqdict_alive_page_int
          <= \_zqdict_entry_get_page:n {##1}
        }
      )
    } {
      \str_if_eq:VVTF \l_zqdict_current_section_str \l_zqdict_previous_section_str {
        % non-first element in a section
        \tl_if_empty:NTF \l_zqdict_buffer_part_tl {
          \tl_set:Nx \l_zqdict_buffer_part_tl {
            { \_zqdict_section_get_options:V \l_tmpa_tl }
            { \_zqdict_entry_get_headword:V \l_zqdict_current_entry_tl }
          }
        } {
          \tl_put_right:Nx \l_zqdict_buffer_part_tl {
            {\_zqdict_entry_get_headword:V \l_zqdict_current_entry_tl}
          }
        }
      } {
        \str_if_eq:VnTF \l_zqdict_previous_section_str {} {
          % first section
          \tl_if_empty:xF {\_zqdict_section_get_partlabel:V \l_tmpa_tl} {
            \_zqdict_section_get_partlabel:V \l_tmpa_tl
            \quad
          }
        } {
          % output elements for the previous part
          \_zqdict_header_render_elements:xx {
            \tl_head:N \l_zqdict_buffer_part_tl
          } {
            \tl_tail:N \l_zqdict_buffer_part_tl
          }
          \tl_clear:N \l_zqdict_buffer_part_tl

          % non-first section
          \quad
        }
        \textbf{
          {
            \ltjsetparameter{jacharrange={-9}}
            \_zqdict_section_get_number:V \l_tmpa_tl
          }
          \str_use:N \l_zqdict_current_section_str
        }
        \hspace{0.5em}

        % the first element in a section
        \tl_set:Nx \l_zqdict_buffer_part_tl {
          { \_zqdict_section_get_options:V \l_tmpa_tl }
          { \_zqdict_entry_get_headword:V \l_zqdict_current_entry_tl }
        }

        \str_set:NV \l_zqdict_previous_section_str \l_zqdict_current_section_str
      }
    }
    \int_compare:nNnT {\l_zqdict_current_page_int} > {\l_zqdict_alive_page_int} {
      % output elements for the current incomplete part
      \_zqdict_header_render_elements:xx {
        \tl_head:N \l_zqdict_buffer_part_tl
      } {
        \tl_tail:N \l_zqdict_buffer_part_tl
      }
      \tl_clear:N \l_zqdict_buffer_part_tl

      \tl_map_break:n {}
    }
  }

  \tl_if_empty:NF \l_zqdict_buffer_part_tl {
    \_zqdict_header_render_elements:xx {
      \tl_head:N \l_zqdict_buffer_part_tl
    } {
      \tl_tail:N \l_zqdict_buffer_part_tl
    }
    \tl_clear:N \l_zqdict_buffer_part_tl
  }
}

\ExplSyntaxOff

\pagestyle{fancy}
\fancyhead[C]{\small\ShowEntriesForPage{\value{page}}}
\fancyhead[LO,RE]{\small\leftmark}
\fancyhead[RO,LE]{\thepage}
\fancyfoot{}

\titleformat{\part}[display]
{%format
  \bfseries\large%
}{%label
  %第\thesection{}部
}{%sep
  0pt
}{%before-code
  \clearpage
}
\counterwithin*{section}{part}
\titleformat{\section}[hang]
{%format
  \bfseries\large%
}{%label
  % (\thesection)%
}{%sep
  0pt
}{%before-code
  \pagebreak[2]%
}
\titleformat{\subsection}[hang]
{%format
  \bfseries\large%
}{%label
  % (\thesection)%
}{%sep
  0pt
}{%before-code
  \pagebreak[2]%
  \centering%
}
\titlespacing*{\subsection}{0pt}{0.5em plus 0.5em minus 0.2em}{0pt}

\ExplSyntaxOn
\NewDocumentEnvironment{MainGroup}{m m m}{
  \titleformat{\section}[hang]
  {%format
    \bfseries\large%
  }{%label
    \thesection%
  }{%sep
    5pt
  }{%before-code
  }
  \section{#1(#2)}\label{group-#1}%
  \markboth{\thesection #1(#2)}{}
  \begin{multicols}{6}\small
  \tl_map_inline:nn {#3} {
    \noindent
    \tl_item:nn {##1} {2}
    \nobreakspace
    \quelle{
      (p.\nobreakspace \pageref{main-part-\tl_item:nn {##1} {1}})
    }
    \par
  }
  \end{multicols}
}{%
  \clearpage%
  \titleformat{\section}[hang]%
  {%format
    \bfseries\large%
  }{%label
    % (\thesection)%
  }{%sep
    0pt
  }{%before-code
    \pagebreak[2]
  }%
}
\newcommand{\PartHeader}[5]{%
  %\section[#1 #2 #3]{{\ltjsetparameter{jacharrange={-9}}#1}#2\MakeUppercase{#3}}%
  \subsection[#2]{#2}%
  \RecordSection{}{#2}%
  \label{main-part-#1}%
  \nopagebreak[3]%
  \ifthenelse{\equal{#4}{}}{%
    \begin{center}
      #5%
    \end{center}
  }{%
    \begin{center}
      {\small #4}%

      \par

      #5%
    \end{center}
  }%
  \vspace{0.5em plus 0.5em minus 0.2em}%
  \nopagebreak[2]%
}
\newsavebox{\EntryHeadwordBox}
\NewDocumentEnvironment{Entry}{m m m m} {
  \noindent
  \savebox{\EntryHeadwordBox}{\fontseries{b}\fontsize{2\baselineskip}{2\baselineskip}\selectfont#1}%
  \setlength{\EntryDescriptionLineHeight}{2\baselineskip}%
  \setlength{\EntryDescriptionLineIndent}{\wd\EntryHeadwordBox}%
  \setlength{\EntryDescriptionLineLength}{\linewidth-\EntryDescriptionLineHeight-\EntryDescriptionLineIndent}%
  \parshape 3
  \EntryDescriptionLineIndent \EntryDescriptionLineLength
  \EntryDescriptionLineIndent \EntryDescriptionLineLength
  0cm \linewidth%
  %
  \makebox[0pt][r]{%
    \fontseries{b}\fontsize{2\baselineskip}{2\baselineskip}\selectfont%
    \raisebox{-1ex}[0pt][0pt]{\usebox{\EntryHeadwordBox}}%
    \hspace*{0.1em}%
  }%
  \makebox[0pt][l]{%
    \raisebox{-0ex}[0pt][0pt]{%
      \hspace{\EntryDescriptionLineLength+0.1\EntryDescriptionLineHeight}%
      \adjustbox{height=0.9\EntryDescriptionLineHeight,valign=t}{\pxpic[mode=px,colors={k=black,w=white}]{#2}}%
    }%
  }%
  \phantomsection%
  \addcontentsline{toc}{subsubsection}{#1}%
  \label{entry-#3}%
  \RecordEntryBegin{#1}%
  \Position{#4}%
}{%
  \RecordEntryEnd{#1}%
  \vspace{0.25em plus 0.75em minus 0.1em}%
  \pagebreak[2]%
}
\ExplSyntaxOff

\newenvironment{Sound}{\begin{itemize}[nosep,labelindent=0.5em,itemindent=-1em,leftmargin=1.5em]}{\end{itemize}}
\newcommand{\SoundItem}[1]{\item[]{#1:} }

\newenvironment{Sense}{%
  \vspace{0.25em plus 0.25em minus 0.05em}%
  \begin{itemize}[itemsep=0.125em,labelindent=0.5em,itemindent=-1em,leftmargin=1.5em,topsep=0pt,parsep=0pt,partopsep=0pt]\small}{\end{itemize}%
}
\newcommand{\SenseItem}[1]{\item[]{\fSemanticLabel\large\MakeUppercase{\raisebox{-0.125ex}{#1}:}} }

\newcommand{\zyepheng}[1]{\textsf{#1}}

\newcommand{\SoundParts}[1]{{\ltjsetparameter{jacharrange={-9}}#1}}
\newcommand{\SoundPart}[2]{#1#2}
\newcommand{\SoundPartN}[1]{#1}
\newcommand{\SoundPartNI}[1]{#1}

\NewDocumentCommand{\Sikrok}{m m}{#1\textsubscript{#2}}
\NewDocumentCommand{\Unicode}{m}{\texttt{#1}}
\newcommand{\refEntry}[2]{\hyperref[entry-#2]{#1}}

\newcommand{\Position}[1]{\pdfmargincomment{#1}}

\ExplSyntaxOn
\str_new:N \l_zqdict_index_subsection_label_str
\ExplSyntaxOff

\NewDocumentEnvironment{SikrokIndex}{}{
  \setlength{\columnsep}{0.25em}
  \setlength{\columnseprule}{0.4pt}
  \begin{multicols}{5}\small
    \RecordSection{}[rangeElements]{}%
}{
  \end{multicols}
}
\NewDocumentEnvironment{SikrokSection}{m}{
  \begin{center}
    \textbf{#1}%
  \end{center}
}{}

\newcommand\quelle[1]{{%
    \unskip\nobreak\hfil\penalty50
    \hskip0pt\hbox{}\nobreak\hfil{#1}%
    \parfillskip=0pt \finalhyphendemerits=0 \par}}
\NewDocumentCommand{\SikrokEntry}{m m o m}{
  \noindent%
  \RecordEntryBegin{#2}%
  \IfNoValueTF{#3}{%
    #2 #4 \hfill\pageref{entry-#1}%
  }{%
    #2 #4\quelle{→#3~\pageref{entry-#1}}%
  }%
  \RecordEntryEnd{#2}%
  \par
}

\NewDocumentEnvironment{UnicodeIndex}{}{
  \setlength{\columnsep}{0.25em}
  \setlength{\columnseprule}{0.4pt}
  \begin{multicols}{7}\small
    \RecordSection{}[rangeElements]{}%
}{
  \end{multicols}
}
\NewDocumentEnvironment{UnicodeSection}{m}{
  \begin{center}
    \textbf{#1}%
  \end{center}
}{}
\NewDocumentCommand{\UnicodeEntry}{m m o m}{
  \noindent%
    \RecordEntryBegin{\Unicode{#2}}%
  \IfNoValueTF{#3}{%
    \Unicode{#2} #4 \hfill\pageref{entry-#1}%
  }{%
    \Unicode{#2} #4\quelle{→#3~\pageref{entry-#1}}%
  }%
    \RecordEntryEnd{\Unicode{#2}}%
  \par
}

\ExplSyntaxOn

\NewDocumentEnvironment{GroupIndex}{}{
  \setlength{\columnsep}{0.25em}
  \setlength{\columnseprule}{0.4pt}
  %\begin{multicols}{2}
  \small
  \begin{hangparas}{2em}{1}
  }{
  \end{hangparas}
  %\end{multicols}
}
\NewDocumentCommand{\GroupIndexEntry}{m m m}{
  \textbf{\ref{group-#1} #1(#2)} (p. \pageref{group-#1}--)
  \tl_map_inline:Nn {#3} {
    \hspace{0.5em plus 0.2em minus 0.1em} \mbox{\tl_item:nn {##1} {2}}
  }
  \par
}

\NewDocumentCommand{\KanjiCdot}{}{
  \raisebox{0.25ex}{\cdot}
}
\NewDocumentEnvironment{RadicalIndexRadicalIndex}{}{
\bgroup
\footnotesize
\begin{itemize}
}{
\end{itemize}
\egroup
}
\NewDocumentCommand{\RadicalIndexRadicalIndexLabel}{m}{
  \item[\textbf{#1畫}]
}
\NewDocumentCommand{\RadicalIndexRadicalIndexElem}{m m}{
  \mbox{#1 ~ #2} \quad
}
\NewDocumentEnvironment{RadicalIndex}{m}{
  \setlength{\columnsep}{0.25em}
  \setlength{\columnseprule}{0.4pt}
  \begin{multicols}{2}\small
  #1
  \end{multicols}
  \setlength{\columnsep}{0.25em}
  \setlength{\columnseprule}{0.4pt}
  \begin{multicols}{8}\small
    \RecordSection{}[rangeElements]{}%
  }{
  \end{multicols}
}
\NewDocumentEnvironment{RadicalSection}{m m}{
  \begin{center}
    \textbf{{\footnotesize #1}\\#2部}\label{index-radical-section-#1-#2}%
    \RecordEntryBegin{#1 #2部}%
  \end{center}
  \str_set:Nn \l_zqdict_index_subsection_label_str {0}
}{
    \RecordEntryEnd{#1 #2部}%
}
\NewDocumentCommand{\RadicalEntry}{m m m m}{
  \noindent
  \str_if_eq:VnF \l_zqdict_index_subsection_label_str {#3} {
    \centering{\footnotesize +#3畫}\\
  }
  \str_set:Nn \l_zqdict_index_subsection_label_str {#3}
  #4~\hfill\pageref{entry-#1}
  \par
}

\NewDocumentEnvironment{ReadingIndex}{m}{
  \setlength{\columnsep}{0.25em}
  \setlength{\columnseprule}{0.4pt}
  \begin{multicols}{7}\small
    \str_clear:N \l_zqdict_index_subsection_label_str
    \RecordSection{}[rangeElements]{}%
}{
  \end{multicols}
}
\NewDocumentEnvironment{ReadingSection}{m o}{
  \begin{center}
    {\fReadingIndexLabel\textbf{#1}}\label{index-reading-section-#1}
    \tl_if_empty:nF {#2} {
      {\footnotesize #2}
    }
  \end{center}
  \str_clear:N \l_zqdict_index_subsection_label_str
}{}
\NewDocumentCommand{\ReadingEntry}{m m m}{
  \noindent
  \RecordEntryBegin{#2}%
  \str_if_eq:VnF \l_zqdict_index_subsection_label_str {#2} {
    \centering #2\\
  }
  \str_set:Nn \l_zqdict_index_subsection_label_str {#2}
  #3 \hfill\pageref{entry-#1}
  \RecordEntryEnd{#2}%
  \par
}
\ExplSyntaxOff

\newcommand{\Book}[1]{{#1}}


\begin{document}
\part{Examples}
やること
\begin{itemize}
\item{} 見出し
\item 字形
\item{} 反切
\item{} 字音欄
  \begin{itemize}
  \item{} 字音欄空白でエラー
  \end{itemize}
\item{} 義
  \begin{itemize}
  \item できたら:括弧の対応を見る(同一の括弧が直接親子にあるのはだめ)
  \end{itemize}
\item{} フォント
  \begin{itemize}
  \item{} 本文書体
    \begin{itemize}
    \item{} Extra Light
    \end{itemize}
  \end{itemize}
\item{} データ
  \begin{itemize}
  \item{} 状態をオプションで取る
  \end{itemize}
\item{} 索引
  \begin{itemize}
  \item{} 全体
    \begin{itemize}
    \item{} 各索引前改頁
    \item{} 重複する見出しを纏める
    \item{} 頁数削減
    \end{itemize}
  \item{} 四角
  \item{} 部首索引
    \begin{itemize}
    \item 部首索引の前に部首一覧
    \end{itemize}
  \item{} 字音索引
  \end{itemize}
\item{} 改頁
  \begin{itemize}
  \item{} 以下の改頁はできるだけ防ぐ。
    \begin{itemize}
    \item{} 節見出し中の改頁
    \item{} 節見出し直後での改頁
    \item{} 親字直後の改頁
    \end{itemize}
  \end{itemize}
\item{} 紙面
  \begin{itemize}
  \item{} 余白の調整
  \end{itemize}
\end{itemize}

\PartHeader{27}{冬聲}{toŋ}{}{\refEntry{冬}{27-冬-test}\refEntry{佟}{27+23-佟-test}{疼}{柊}{鼕}{螽}{終}}

\begin{Entry}{冬}{{k}}{27-冬-test}
  + 0(夊部0畫)
  \\
  \Sikrok{2730}{3}
  \begin{Sound}
    \SoundItem{toŋ}《P3798》都/宗[ha]反,端-冬平,上平聲二冬
  \end{Sound}
\end{Entry}

\begin{Entry}{佟}{{k}}{27+23-佟-test}
  + 23人(人部5畫)
  \\
  \Sikrok{2723}{3}
  \begin{Sound}
    \SoundItem{doŋ}《P3798》徒冬反,定冬平
  \end{Sound}
\end{Entry}

\PartHeader{28}{中聲}{tuŋ}{}{\refEntry{中}{28-中-test}仲忠}

\begin{Entry}{中}{{w}}{28-中-test}%
  + 0。丨部3畫。
  \\
  \Sikrok{5000}{6}。
  \begin{Sound}
    \SoundItem{tryuŋ}《切二》陟隆反, 《廣韵》陟隆切, 知東平。上平聲一東(中間)。
    \SoundItem{tryûŋ}《P3696(1)r》陟仲反, 知送去。去聲一送(射中, 撃中)。
    \SoundItem{dryûŋ}《集韵》直衆切
  \end{Sound}
  \begin{Sense}
    \SenseItem{tryuŋ}義略。
    \SenseItem{tryûŋ}義略。
  \end{Sense}
\end{Entry}

\cleardoublepage{}
\part{Generated}
\input{test-data-main}
\RestorePageLayout

\part{Indices}
\part{四角號碼索引}\label{part:四角號碼索引}
\markboth{四角號碼索引}{}
\input{test-data-index-sikrok}
\part{《漢辭海》部首索引}
\markboth{《漢辭海》部首索引}{}
\input{test-data-index-radical}
\part{隋拼索引}
\markboth{隋拼索引}{}
\input{test-data-index-reading}
\part{Unicode索引}\label{part:Unicode索引}
\markboth{Unicode索引}{}
\input{test-data-index-unicode}
\part{諧聲部目録}\label{part:諧聲部目録}
\markboth{諧聲部目録}{}
\input{test-data-index-groups}
\part{《玉篇》部首一覽}\label{part:玉篇部首一覽}
\markboth{《玉篇》部首一覽}{}
\begin{PartRadicals}
\PartRadicalEntry{一}{1}
\PartRadicalEntry{上}{2}
\PartRadicalEntry{示}{3}
\PartRadicalEntry{二}{4}
\PartRadicalEntry{三}{5}
\PartRadicalEntry{王}{6}
\PartRadicalEntry{玉}{7}
\PartRadicalEntry{玨}{8}
\PartRadicalEntry{土}{9}
\PartRadicalEntry{垚}{10}
\PartRadicalEntry{堇}{11}
\PartRadicalEntry{里}{12}
\PartRadicalEntry{田}{13}
\PartRadicalEntry{畕}{14}
\PartRadicalEntry{黄}{15}
\PartRadicalEntry{丘}{16}
\PartRadicalEntry{京}{17}
\PartRadicalEntry{冂}{18}
\PartRadicalEntry{𩫏}{19}
\PartRadicalEntry{邑}{20}
\PartRadicalEntry{司}{21}
\PartRadicalEntry{士}{22}
\PartRadicalEntry{人}{23}
\PartRadicalEntry{儿}{24}
\PartRadicalEntry{父}{25}
\PartRadicalEntry{臣}{26}
\PartRadicalEntry{男}{27}
\PartRadicalEntry{民}{28}
\PartRadicalEntry{夫}{29}
\PartRadicalEntry{予}{30}
\PartRadicalEntry{我}{31}
\PartRadicalEntry{身}{32}
\PartRadicalEntry{兄}{33}
\PartRadicalEntry{弟}{34}
\PartRadicalEntry{女}{35}
\PartRadicalEntry{頁}{36}
\PartRadicalEntry{頻}{37}
\PartRadicalEntry{𦣻}{38}
\PartRadicalEntry{首}{39}
\PartRadicalEntry{県}{40}
\PartRadicalEntry{面}{41}
\PartRadicalEntry{色}{42}
\PartRadicalEntry{囟}{43}
\PartRadicalEntry{𦣝}{44}
\PartRadicalEntry{亢}{45}
\PartRadicalEntry{鼻}{46}
\PartRadicalEntry{自}{47}
\PartRadicalEntry{𦣹}{48}
\PartRadicalEntry{目}{49}
\PartRadicalEntry{盾}{50}
\PartRadicalEntry{䀠}{51}
\PartRadicalEntry{𥄎}{52}
\PartRadicalEntry{見}{53}
\PartRadicalEntry{覞}{54}
\PartRadicalEntry{𥄕}{55}
\PartRadicalEntry{耳}{56}
\PartRadicalEntry{口}{57}
\PartRadicalEntry{𧮫}{58}
\PartRadicalEntry{舌}{59}
\PartRadicalEntry{齒}{60}
\PartRadicalEntry{牙}{61}
\PartRadicalEntry{須}{62}
\PartRadicalEntry{彡}{63}
\PartRadicalEntry{彣}{64}
\PartRadicalEntry{文}{65}
\PartRadicalEntry{髟}{66}
\PartRadicalEntry{手}{67}
\PartRadicalEntry{廾}{68}
\PartRadicalEntry{𠬜}{69}
\PartRadicalEntry{舁}{70}
\PartRadicalEntry{𦥑}{71}
\PartRadicalEntry{爪}{72}
\PartRadicalEntry{丮}{73}
\PartRadicalEntry{鬥}{74}
\PartRadicalEntry{𠂇}{75}
\PartRadicalEntry{又}{76}
\PartRadicalEntry{足}{77}
\PartRadicalEntry{疋}{78}
\PartRadicalEntry{乖}{79}
\PartRadicalEntry{骨}{80}
\PartRadicalEntry{血}{81}
\PartRadicalEntry{肉}{82}
\PartRadicalEntry{筋}{83}
\PartRadicalEntry{力}{84}
\PartRadicalEntry{劦}{85}
\PartRadicalEntry{吕}{86}
\PartRadicalEntry{㝱}{87}
\PartRadicalEntry{心}{88}
\PartRadicalEntry{思}{89}
\PartRadicalEntry{惢}{90}
\PartRadicalEntry{言}{91}
\PartRadicalEntry{誩}{92}
\PartRadicalEntry{曰}{93}
\PartRadicalEntry{乃}{94}
\PartRadicalEntry{丂}{95}
\PartRadicalEntry{可}{96}
\PartRadicalEntry{兮}{97}
\PartRadicalEntry{号}{98}
\PartRadicalEntry{于}{99}
\PartRadicalEntry{云}{100}
\PartRadicalEntry{音}{101}
\PartRadicalEntry{告}{102}
\PartRadicalEntry{凵}{103}
\PartRadicalEntry{吅}{104}
\PartRadicalEntry{品}{105}
\PartRadicalEntry{喿}{106}
\PartRadicalEntry{龠}{107}
\PartRadicalEntry{册}{108}
\PartRadicalEntry{㗊}{109}
\PartRadicalEntry{只}{110}
\PartRadicalEntry{㕯}{111}
\PartRadicalEntry{欠}{112}
\PartRadicalEntry{食}{113}
\PartRadicalEntry{甘}{114}
\PartRadicalEntry{旨}{115}
\PartRadicalEntry{㳄}{116}
\PartRadicalEntry{㚔}{117}
\PartRadicalEntry{夲}{118}
\PartRadicalEntry{夰}{119}
\PartRadicalEntry{彳}{120}
\PartRadicalEntry{行}{121}
\PartRadicalEntry{冘}{122}
\PartRadicalEntry{夂}{123}
\PartRadicalEntry{久}{124}
\PartRadicalEntry{夊}{125}
\PartRadicalEntry{舛}{126}
\PartRadicalEntry{走}{127}
\PartRadicalEntry{辵}{128}
\PartRadicalEntry{廴}{129}
\PartRadicalEntry{癶}{130}
\PartRadicalEntry{步}{131}
\PartRadicalEntry{止}{132}
\PartRadicalEntry{處}{133}
\PartRadicalEntry{立}{134}
\PartRadicalEntry{並}{135}
\PartRadicalEntry{此}{136}
\PartRadicalEntry{正}{137}
\PartRadicalEntry{是}{138}
\PartRadicalEntry{宀}{139}
\PartRadicalEntry{宫}{140}
\PartRadicalEntry{宁}{141}
\PartRadicalEntry{門}{142}
\PartRadicalEntry{户}{143}
\PartRadicalEntry{尸}{144}
\PartRadicalEntry{尾}{145}
\PartRadicalEntry{尺}{146}
\PartRadicalEntry{履}{147}
\PartRadicalEntry{老}{148}
\PartRadicalEntry{疒}{149}
\PartRadicalEntry{𣦼}{150}
\PartRadicalEntry{歹}{151}
\PartRadicalEntry{死}{152}
\PartRadicalEntry{冎}{153}
\PartRadicalEntry{凶}{154}
\PartRadicalEntry{穴}{155}
\PartRadicalEntry{丨}{156}
\PartRadicalEntry{屮}{157}
\PartRadicalEntry{木}{158}
\PartRadicalEntry{林}{159}
\PartRadicalEntry{巢}{160}
\PartRadicalEntry{叒}{161}
\PartRadicalEntry{艸}{162}
\PartRadicalEntry{蓐}{163}
\PartRadicalEntry{茻}{164}
\PartRadicalEntry{舜}{165}
\PartRadicalEntry{竹}{166}
\PartRadicalEntry{箕}{167}
\PartRadicalEntry{才}{168}
\PartRadicalEntry{𣎵}{169}
\PartRadicalEntry{乇}{170}
\PartRadicalEntry{𠂹}{171}
\PartRadicalEntry{𠌶}{172}
\PartRadicalEntry{華}{173}
\PartRadicalEntry{𥝌}{174}
\PartRadicalEntry{稽}{175}
\PartRadicalEntry{桼}{176}
\PartRadicalEntry{丵}{177}
\PartRadicalEntry{菐}{178}
\PartRadicalEntry{龴}{179}
\PartRadicalEntry{𣐺}{180}
\PartRadicalEntry{𠧪}{181}
\PartRadicalEntry{朿}{182}
\PartRadicalEntry{𣎳}{183}
\PartRadicalEntry{𣏟}{184}
\PartRadicalEntry{麻}{185}
\PartRadicalEntry{尗}{186}
\PartRadicalEntry{韭}{187}
\PartRadicalEntry{瓜}{188}
\PartRadicalEntry{瓠}{189}
\PartRadicalEntry{丯}{190}
\PartRadicalEntry{耒}{191}
\PartRadicalEntry{來}{192}
\PartRadicalEntry{麥}{193}
\PartRadicalEntry{黍}{194}
\PartRadicalEntry{禾}{195}
\PartRadicalEntry{秝}{196}
\PartRadicalEntry{香}{197}
\PartRadicalEntry{皀}{198}
\PartRadicalEntry{鬯}{199}
\PartRadicalEntry{米}{200}
\PartRadicalEntry{𥽿}{201}
\PartRadicalEntry{臼}{202}
\PartRadicalEntry{倉}{203}
\PartRadicalEntry{㐭}{204}
\PartRadicalEntry{嗇}{205}
\PartRadicalEntry{亼}{206}
\PartRadicalEntry{會}{207}
\PartRadicalEntry{享}{208}
\PartRadicalEntry{㫗}{209}
\PartRadicalEntry{畐}{210}
\PartRadicalEntry{入}{211}
\PartRadicalEntry{冖}{212}
\PartRadicalEntry{𠔼}{213}
\PartRadicalEntry{冃}{214}
\PartRadicalEntry{㒳}{215}
\PartRadicalEntry{㡀}{216}
\PartRadicalEntry{襾}{217}
\PartRadicalEntry{网}{218}
\PartRadicalEntry{𠦒}{219}
\PartRadicalEntry{冓}{220}
\PartRadicalEntry{𠙴}{221}
\PartRadicalEntry{去}{222}
\PartRadicalEntry{北}{223}
\PartRadicalEntry{西}{224}
\PartRadicalEntry{鹵}{225}
\PartRadicalEntry{鹽}{226}
\PartRadicalEntry{𡈼}{227}
\PartRadicalEntry{重}{228}
\PartRadicalEntry{卧}{229}
\PartRadicalEntry{㐆}{230}
\PartRadicalEntry{琴}{231}
\PartRadicalEntry{喜}{232}
\PartRadicalEntry{壴}{233}
\PartRadicalEntry{鼓}{234}
\PartRadicalEntry{豈}{235}
\PartRadicalEntry{豆}{236}
\PartRadicalEntry{豊}{237}
\PartRadicalEntry{豐}{238}
\PartRadicalEntry{䖒}{239}
\PartRadicalEntry{皿}{240}
\PartRadicalEntry{鼎}{241}
\PartRadicalEntry{瓦}{242}
\PartRadicalEntry{缶}{243}
\PartRadicalEntry{鬲}{244}
\PartRadicalEntry{䰜}{245}
\PartRadicalEntry{斗}{246}
\PartRadicalEntry{勺}{247}
\PartRadicalEntry{几}{248}
\PartRadicalEntry{且}{249}
\PartRadicalEntry{匚}{250}
\PartRadicalEntry{曲}{251}
\PartRadicalEntry{壷}{252}
\PartRadicalEntry{巵}{253}
\PartRadicalEntry{甾}{254}
\PartRadicalEntry{㫃}{255}
\PartRadicalEntry{勿}{256}
\PartRadicalEntry{矢}{257}
\PartRadicalEntry{弓}{258}
\PartRadicalEntry{弜}{259}
\PartRadicalEntry{斤}{260}
\PartRadicalEntry{矛}{261}
\PartRadicalEntry{戈}{262}
\PartRadicalEntry{殳}{263}
\PartRadicalEntry{殺}{264}
\PartRadicalEntry{戉}{265}
\PartRadicalEntry{刀}{266}
\PartRadicalEntry{㓞}{267}
\PartRadicalEntry{刃}{268}
\PartRadicalEntry{金}{269}
\PartRadicalEntry{攴}{270}
\PartRadicalEntry{放}{271}
\PartRadicalEntry{丌}{272}
\PartRadicalEntry{左}{273}
\PartRadicalEntry{工}{274}
\PartRadicalEntry{㠭}{275}
\PartRadicalEntry{巫}{276}
\PartRadicalEntry{卜}{277}
\PartRadicalEntry{兆}{278}
\PartRadicalEntry{用}{279}
\PartRadicalEntry{爻}{280}
\PartRadicalEntry{㸚}{281}
\PartRadicalEntry{車}{282}
\PartRadicalEntry{舟}{283}
\PartRadicalEntry{方}{284}
\PartRadicalEntry{水}{285}
\PartRadicalEntry{沝}{286}
\PartRadicalEntry{𡿨}{287}
\PartRadicalEntry{巜}{288}
\PartRadicalEntry{川}{289}
\PartRadicalEntry{井}{290}
\PartRadicalEntry{泉}{291}
\PartRadicalEntry{灥}{292}
\PartRadicalEntry{永}{293}
\PartRadicalEntry{𠂢}{294}
\PartRadicalEntry{谷}{295}
\PartRadicalEntry{冫}{296}
\PartRadicalEntry{雨}{297}
\PartRadicalEntry{雲}{298}
\PartRadicalEntry{風}{299}
\PartRadicalEntry{气}{300}
\PartRadicalEntry{鬼}{301}
\PartRadicalEntry{甶}{302}
\PartRadicalEntry{白}{303}
\PartRadicalEntry{日}{304}
\PartRadicalEntry{旦}{305}
\PartRadicalEntry{晨}{306}
\PartRadicalEntry{乾}{307}
\PartRadicalEntry{晶}{308}
\PartRadicalEntry{月}{309}
\PartRadicalEntry{有}{310}
\PartRadicalEntry{明}{311}
\PartRadicalEntry{冏}{312}
\PartRadicalEntry{冥}{313}
\PartRadicalEntry{夕}{314}
\PartRadicalEntry{多}{315}
\PartRadicalEntry{小}{316}
\PartRadicalEntry{幺}{317}
\PartRadicalEntry{𢆶}{318}
\PartRadicalEntry{玄}{319}
\PartRadicalEntry{丏}{320}
\PartRadicalEntry{奢}{321}
\PartRadicalEntry{大}{322}
\PartRadicalEntry{火}{323}
\PartRadicalEntry{炎}{324}
\PartRadicalEntry{焱}{325}
\PartRadicalEntry{炙}{326}
\PartRadicalEntry{爨}{327}
\PartRadicalEntry{匆}{328}
\PartRadicalEntry{黑}{329}
\PartRadicalEntry{赤}{330}
\PartRadicalEntry{亦}{331}
\PartRadicalEntry{夨}{332}
\PartRadicalEntry{夭}{333}
\PartRadicalEntry{交}{334}
\PartRadicalEntry{尢}{335}
\PartRadicalEntry{壹}{336}
\PartRadicalEntry{專}{337}
\PartRadicalEntry{丶}{338}
\PartRadicalEntry{丹}{339}
\PartRadicalEntry{青}{340}
\PartRadicalEntry{氏}{341}
\PartRadicalEntry{氐}{342}
\PartRadicalEntry{山}{343}
\PartRadicalEntry{屾}{344}
\PartRadicalEntry{嵬}{345}
\PartRadicalEntry{屵}{346}
\PartRadicalEntry{广}{347}
\PartRadicalEntry{厂}{348}
\PartRadicalEntry{高}{349}
\PartRadicalEntry{危}{350}
\PartRadicalEntry{石}{351}
\PartRadicalEntry{磬}{352}
\PartRadicalEntry{𠂤}{353}
\PartRadicalEntry{阜}{354}
\PartRadicalEntry{𨺅}{355}
\PartRadicalEntry{厽}{356}
\PartRadicalEntry{馬}{357}
\PartRadicalEntry{牛}{358}
\PartRadicalEntry{氂}{359}
\PartRadicalEntry{羊}{360}
\PartRadicalEntry{羴}{361}
\PartRadicalEntry{𦫳}{362}
\PartRadicalEntry{萈}{363}
\PartRadicalEntry{犬}{364}
\PartRadicalEntry{㹜}{365}
\PartRadicalEntry{豕}{366}
\PartRadicalEntry{豚}{367}
\PartRadicalEntry{㣇}{368}
\PartRadicalEntry{彑}{369}
\PartRadicalEntry{嘼}{370}
\PartRadicalEntry{廌}{371}
\PartRadicalEntry{鹿}{372}
\PartRadicalEntry{麤}{373}
\PartRadicalEntry{㲋}{374}
\PartRadicalEntry{兔}{375}
\PartRadicalEntry{厹}{376}
\PartRadicalEntry{兕}{377}
\PartRadicalEntry{象}{378}
\PartRadicalEntry{能}{379}
\PartRadicalEntry{熊}{380}
\PartRadicalEntry{龍}{381}
\PartRadicalEntry{虍}{382}
\PartRadicalEntry{虎}{383}
\PartRadicalEntry{虤}{384}
\PartRadicalEntry{豸}{385}
\PartRadicalEntry{烏}{386}
\PartRadicalEntry{𠘧}{387}
\PartRadicalEntry{𠃉}{388}
\PartRadicalEntry{燕}{389}
\PartRadicalEntry{鳥}{390}
\PartRadicalEntry{隹}{391}
\PartRadicalEntry{奞}{392}
\PartRadicalEntry{雈}{393}
\PartRadicalEntry{瞿}{394}
\PartRadicalEntry{雔}{395}
\PartRadicalEntry{雥}{396}
\PartRadicalEntry{魚}{397}
\PartRadicalEntry{𩺰}{398}
\PartRadicalEntry{鼠}{399}
\PartRadicalEntry{易}{400}
\PartRadicalEntry{虫}{401}
\PartRadicalEntry{䖵}{402}
\PartRadicalEntry{蟲}{403}
\PartRadicalEntry{它}{404}
\PartRadicalEntry{龜}{405}
\PartRadicalEntry{黽}{406}
\PartRadicalEntry{卵}{407}
\PartRadicalEntry{貝}{408}
\PartRadicalEntry{羽}{409}
\PartRadicalEntry{飛}{410}
\PartRadicalEntry{習}{411}
\PartRadicalEntry{卂}{412}
\PartRadicalEntry{非}{413}
\PartRadicalEntry{不}{414}
\PartRadicalEntry{至}{415}
\PartRadicalEntry{毛}{416}
\PartRadicalEntry{毳}{417}
\PartRadicalEntry{冉}{418}
\PartRadicalEntry{而}{419}
\PartRadicalEntry{角}{420}
\PartRadicalEntry{皮}{421}
\PartRadicalEntry{㼱}{422}
\PartRadicalEntry{革}{423}
\PartRadicalEntry{韋}{424}
\PartRadicalEntry{糸}{425}
\PartRadicalEntry{系}{426}
\PartRadicalEntry{素}{427}
\PartRadicalEntry{絲}{428}
\PartRadicalEntry{黹}{429}
\PartRadicalEntry{率}{430}
\PartRadicalEntry{索}{431}
\PartRadicalEntry{巾}{432}
\PartRadicalEntry{巿}{433}
\PartRadicalEntry{帛}{434}
\PartRadicalEntry{衣}{435}
\PartRadicalEntry{裘}{436}
\PartRadicalEntry{卩}{437}
\PartRadicalEntry{印}{438}
\PartRadicalEntry{𠨍}{439}
\PartRadicalEntry{辟}{440}
\PartRadicalEntry{茍}{441}
\PartRadicalEntry{勹}{442}
\PartRadicalEntry{包}{443}
\PartRadicalEntry{長}{444}
\PartRadicalEntry{𠤎}{445}
\PartRadicalEntry{匕}{446}
\PartRadicalEntry{从}{447}
\PartRadicalEntry{乑}{448}
\PartRadicalEntry{共}{449}
\PartRadicalEntry{異}{450}
\PartRadicalEntry{史}{451}
\PartRadicalEntry{支}{452}
\PartRadicalEntry{𠬪}{453}
\PartRadicalEntry{帇}{454}
\PartRadicalEntry{聿}{455}
\PartRadicalEntry{書}{456}
\PartRadicalEntry{隶}{457}
\PartRadicalEntry{臤}{458}
\PartRadicalEntry{帀}{459}
\PartRadicalEntry{比}{460}
\PartRadicalEntry{出}{461}
\PartRadicalEntry{之}{462}
\PartRadicalEntry{生}{463}
\PartRadicalEntry{耑}{464}
\PartRadicalEntry{毌}{465}
\PartRadicalEntry{束}{466}
\PartRadicalEntry{㯻}{467}
\PartRadicalEntry{囗}{468}
\PartRadicalEntry{員}{469}
\PartRadicalEntry{齊}{470}
\PartRadicalEntry{干}{471}
\PartRadicalEntry{幵}{472}
\PartRadicalEntry{片}{473}
\PartRadicalEntry{牀}{474}
\PartRadicalEntry{毋}{475}
\PartRadicalEntry{克}{476}
\PartRadicalEntry{录}{477}
\PartRadicalEntry{丿}{478}
\PartRadicalEntry{𠂆}{479}
\PartRadicalEntry{弋}{480}
\PartRadicalEntry{乁}{481}
\PartRadicalEntry{亅}{482}
\PartRadicalEntry{句}{483}
\PartRadicalEntry{丩}{484}
\PartRadicalEntry{乚}{485}
\PartRadicalEntry{亡}{486}
\PartRadicalEntry{匸}{487}
\PartRadicalEntry{兂}{488}
\PartRadicalEntry{旡}{489}
\PartRadicalEntry{皃}{490}
\PartRadicalEntry{𠑹}{491}
\PartRadicalEntry{先}{492}
\PartRadicalEntry{秃}{493}
\PartRadicalEntry{厶}{494}
\PartRadicalEntry{單}{495}
\PartRadicalEntry{四}{496}
\PartRadicalEntry{叕}{497}
\PartRadicalEntry{亞}{498}
\PartRadicalEntry{五}{499}
\PartRadicalEntry{六}{500}
\PartRadicalEntry{七}{501}
\PartRadicalEntry{八}{502}
\PartRadicalEntry{釆}{503}
\PartRadicalEntry{半}{504}
\PartRadicalEntry{九}{505}
\PartRadicalEntry{丸}{506}
\PartRadicalEntry{十}{507}
\PartRadicalEntry{卅}{508}
\PartRadicalEntry{古}{509}
\PartRadicalEntry{寸}{510}
\PartRadicalEntry{丈}{511}
\PartRadicalEntry{皕}{512}
\PartRadicalEntry{甲}{513}
\PartRadicalEntry{乙}{514}
\PartRadicalEntry{丙}{515}
\PartRadicalEntry{丁}{516}
\PartRadicalEntry{戊}{517}
\PartRadicalEntry{己}{518}
\PartRadicalEntry{巴}{519}
\PartRadicalEntry{庚}{520}
\PartRadicalEntry{辛}{521}
\PartRadicalEntry{辡}{522}
\PartRadicalEntry{䇂}{523}
\PartRadicalEntry{桀}{524}
\PartRadicalEntry{壬}{525}
\PartRadicalEntry{癸}{526}
\PartRadicalEntry{子}{527}
\PartRadicalEntry{了}{528}
\PartRadicalEntry{孨}{529}
\PartRadicalEntry{𠫓}{530}
\PartRadicalEntry{丑}{531}
\PartRadicalEntry{寅}{532}
\PartRadicalEntry{卯}{533}
\PartRadicalEntry{辰}{534}
\PartRadicalEntry{巳}{535}
\PartRadicalEntry{午}{536}
\PartRadicalEntry{未}{537}
\PartRadicalEntry{申}{538}
\PartRadicalEntry{酉}{539}
\PartRadicalEntry{酋}{540}
\PartRadicalEntry{戌}{541}
\PartRadicalEntry{亥}{542}
\end{PartRadicals}


\end{document}
